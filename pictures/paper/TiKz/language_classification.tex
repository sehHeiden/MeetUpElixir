% !TeX spellcheck = en_GB

% Define block styles
\tikzset{
	block/.style={
		rectangle, 
		draw, 
		fill=blue!20, 
		text width=15em, 
		text centered, 
		rounded corners, 
		minimum height=3em, 
		drop shadow
	},
	doubleblock/.style={ % New style for double-lined blocks
		rectangle, 
		draw=black, 
		double, 
		double distance=2pt, % Adjust the distance between the lines here
		fill=blue!20, 
		text width=15em, 
		text centered, 
		rounded corners, 
		minimum height=3em, 
		drop shadow
	},
	line/.style={
		draw, 
		-{Latex[length=3mm,width=3mm]}, 
		line width=0.7mm,
	}
}



\begin{tikzpicture}[node distance=3.0cm and 2cm, auto, every node/.style={font=\Huge}] % Adjusted node distance for wider separation
	
	\coordinate (start) at (0, 1cm);
	% Nodes
	\node [block, below=1cm of start] (init) {XLM-RoBERTa};
	%\node [block, below left=2cm and 3cm of init] (german) {German}; % Position adjusted for wider layout
	%\node [block, below right=2cm and 3cm of init] (english) {English}; % Position adjusted for wider layout
	\node [block, below left=3cm and 3cm of init] (germanbert) {german-sentiment\_bert};
	\node [block, below right=3cm and 3cm of init] (roberta) {RoBERTa BERTweet - Sentiment};
	\node [block, below=3cm of germanbert] (bavarianq1) {Bavarian?};
	\node [block, below=3cm of roberta] (bavarianq2) {Bavarian?};
	
	\node [doubleblock, below=3cm of bavarianq1] (german2) {German}; % Position adjusted for separation
	\node [doubleblock, below right=3cm and 1cm of bavarianq1] (bavarian) {Bavarian}; % Position adjusted for separation
	\node [doubleblock] at ([yshift=-2.5cm]bavarianq2.south) (english2) {English}; % Adjusted position


	\draw [line] (start) -- (bavinstance);
	
	% Edges
	\path [line] (init) -| node [near start] {German} (germanbert);
	\path [line] (init) -| node [near start] {English} (roberta);
	%\path [line] (german) -- (germanbert);
	%\path [line] (english) -- (roberta);
	\path [line] (germanbert) -- (bavarianq1);
	\path [line] (roberta) -- (bavarianq2);
	\path [line] (bavarianq1) -- node [near start, right] {yes} (bavarian);
	\path [line] (bavarianq1) -- node [near start, left] {no} (german2);
	\path [line] (bavarianq2) -- node [near start, left] {yes} (bavarian);
	\path [line] (bavarianq2) -- node [near start, right] {no} (english2);
\end{tikzpicture}
